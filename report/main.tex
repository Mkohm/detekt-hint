\documentclass{article}
\usepackage[utf8]{inputenc}
\usepackage[acronym]{glossaries}
\usepackage{biblatex}
\usepackage[colorinlistoftodos]{todonotes}
\usepackage{graphicx}
\usepackage{pgfplots}
\usepackage{fancyvrb}

\addbibresource{bibliography.bib}


\title{Detekt-hint -- Detection of programming principle violations}
\author{Marius Kohmann}
\date{June 2020}


\newacronym{ast}{AST}{Abstract Syntax Tree}
\newacronym{solid}{SOLID}{\textbf{S}ingle responsibility, \textbf{O}pen–closed, \textbf{L}iskov substitution, \textbf{I}nterface segregation, \textbf{D}ependency inversion}
\newacronym{dry}{DRY}{Don't Repeat Yourself}
\newacronym{srp}{SRP}{Single Responsibility Principle}
\newacronym{ocp}{OCP}{Open-Closed Principle}
\newacronym{lsp}{LSP}{Liskov Substitution Principle}
\newacronym{isp}{ISP}{Interface Segregation Principle}
\newacronym{dip}{DIP}{Dependency Inversion Principle}
\newacronym{lcom}{LCOM}{Lack of Cohesion Of Methods}

\begin{document}

\maketitle

\begin{abstract}
	%Sample IMRaD abstract: 
	
	
	
\end{abstract}


\clearpage
\tableofcontents
\clearpage
\section{Introduction}
Writing maintainable and high quality code is an important part of software engineering. Ignoring quality enhancing tasks like refactoring\footnote{From Wikipedia: Process of restructuring existing code without changing its external behavior \cite{refactoring}.} and testing will introduce technical debt\footnote{From Wikipedia: Concept in software development that reflects the implied cost of additional rework caused by choosing an easy (limited) solution now instead of using a better approach that would take longer \cite{technicalDebt}.}, that will have negative impact on further development. Some consequences include increased development time, inaccurate estimations causing lost deadlines and higher costs of introducing new developers to the project.
Various people define software maintainability differently, but a commonly
accepted collection of quality attributes include extensibility, modularity, testability and understandability. To further help developers write code with those attributes, rules, conventions and programming principles have been developed. For developers it is a tough challenge to fulfill all of the mentioned attributes of quality code and therefore tools for code analysis have been created. By using code analysis one is able to detect and resolve problems, before the problems arise in a production environment. In a preliminary study on the state of the art of tools for code analysis one of the main takeaways was that there were no tools focusing on detecting violations of programming principles. 

%Objective
The objective of this study is to develop knowledge about detection of programming principle violations using the design science methodology. The design science methodology involves development of an innovative product through iterations including prototypes and continuous feedback of the product. The and a detailed evaluation of the final artifact.

The outline of the paper is as follows; Section \ref{background} gives some background information and an introduction to the topics of writing code with high quality and code analysis. Section \ref{methodology} describes the methodology of the research process. Section \ref{relatedwork} gives some insight in related work in the area and section \ref{results} presents the developed artifact, and a detailed evaluation of it. Section \ref{discussion} discusses the results, and lastly in section \ref{conclusion} we conclude the paper.  

\section{Background}
\label{background}
To be able to develop a tool for detection of programming principle violations using the design science methodology there are a handful of \todo{better word} subjects you need to be familiar with. The following sections will give a brief introduction to them.


\subsection{Code analysis}
\todo{rewrite this to fit programming principles and PSI}
To help developers of software systems adhere to these concepts and conventions tools for code analysis have been developed. In code analysis we differentiate between two types of code analysis, dynamic code analysis and static code analysis. \textit{Dynamic code analysis} is done by analysing programs being executed on a processor, while \textit{static code analysis} is purely based on analysis of the source code. Since dynamic analysis is based on program execution it has the advantage of being able to measure the actual CPU, memory and energy performance of the application. 

%It can also gather information about the system that would be hard to detect using static code analysis. Examples include discovery of dynamic dependencies using reflection, or actual application usage. However, that does not mean that static code analysis is not able to target performance or dynamic aspects of source code.

As we search to improve the source code, we have chosen to focus on static code analysis. The static analysis is done by parsing the source code, creating an \gls{ast} and then analyzing the tree for violations of the aforementioned concepts and conventions. Figure \ref{fig:ast} shows a simple example \gls{ast} where a static analysis tools could detect that the expression \texttt{x == 1} always evaluates to \texttt{true} and that the variable \texttt{y} is never used. The tool could then suggest that the branching is unnecessary and that the \texttt{y} variable is removed.  

\todo{Exchange this for an example with useCompositionInsteadOfInheritance rule}
\begin{figure}[h!]
	\centering
	\includegraphics[width=\linewidth/2]{report/ast.png}
	\caption{Simple \gls{ast} of x=1; if (x == 1) \{y = 10;\}}
	\label{fig:ast}
\end{figure}


\subsection{Design science}
Design science is a research methodology that focuses on getting knowledge through development of innovative artifacts. The methodology provides specific guidelines for evaluation and iteration in research projects.
\todo{insert image of design science methodology}

\subsection{Prototyping}

horizontal and vertical prototypes. high and low res prototypes.

\subsection{Design principles}
Design principles (also called programming principles) are a set of guidelines that programmers should follow to avoid bad design. According to Robert C. Martin \cite{robertcmartinprinciples} there are three characteristics of bad design that the design principles will help reduce:

\begin{enumerate}
	\item Rigidity - It is hard to change because every change affects too many other parts of the system.
	\item Fragility - When you make a change, unexpected parts of the system break.
	\item Immobility - It is hard to reuse in another application because it cannot be disentangled from the current application.
\end{enumerate}

A common set of design principles that often is referred to is the \gls{solid} principles and \gls{dry} principle. The design principles are often abstract and verification requires knowledge and reasoning about the business domain. They are therefore hard to verify using code analysis. 

\todo{In depth description of SOLID principles as they are important in the tool}


One could easily think that \gls{dry} could be verified through looking at similar code, but it turns out that it is not that simple. Two code snippets could be exact copies of each other, but still not violating \gls{dry} if the two snippets would change for different reasons. 

This is the core problem i am facing in this study - helping the developers with these decisions even if the problem could not be verified.


\subsection{False positives}
As stated above, design principles are hard to verify and is the reason why very few tools cover them. Because we could not be 100\% sure if a design principle is followed we would end up with our tool reporting issues, that in reality are not issues. We call these \textit{false-positives}. These create noise and are obtrusive for the developer, so it is important to keep the false-positive rate as low as possible.


\subsection{MVP}

\section{Related work}
\label{relatedwork}
Ndepend. Jarchitect. 

\section{Methodology}
\label{methodology}
This study will follow the Design Science methodology which is presented in (A Design Science Research Methodology for Information Systems Research) \cite{10.2753/MIS0742-1222240302}. It is selected because it fits the process of building a innovative software artifact. The software artifact will be created through a series of iterations that include the follow activities; problem identification/understanding and motivation, definition of objectives for a solution, design and development, demonstration/testing, evaluation, communication.  An illustration of the design science methodology can be seen in figure \ref{fig:designScience}. As can be seen in the figure is important to notice that the activities is not done in any particular order, but in that order as seen required.  \todo{Should i create my own figure that includes problem understanding? Or is this not part of design science? I would then argue that i slight modification of the method should be used that includes iterations on the first step.}. I will elaborate on how the different activities is applied to the project below. 

\subsection{Problem understanding, identification and motivation}

The problem is that very few tools for programming principle violations exists. The existing tools analyze the whole codebase and can be used to find where to start reducing technical debt, but is not highly integrated into the development process and asks the questions if the newly developed code is good - while the developer is in the context. The developer can do a consoucious decision to do refactoring before the code is merged in. 

Current tools (linters) have low false-positive rate, but are not able to target important things like the architecture of the application. This tool will be able to target the architecture, but with higher false-positive rate, and through the format of QA hopefully not disturbing the development process too much. 


The motivation is that programming principles are important because they create structure and lay the foundations of building and extending the product.

based on the pre study and own needs i decided to build the tool.

Ok to reference popularity on github/reddit, but can be misvisende and it is important to not take the response there for god fisk.

\subsection{Define objectives of a solution}
Involves creating a specification of the solution.

\subsection{Design and development}
Involves the development of the final software artifact and both horizontal and vertical prototypes. Prototypes is used early in the development phases to quickly be able to evaluate the product. Horizontal prototypes should be created to early demonstrate a broad view of the final artifact. Vertical prototypes will be used to elaborate on technical challenges and the workings of specific features. 
 
\subsection{Demonstration}
Gather people for an workshop? 
 
\subsection{Evaluation}
I find that it is super hard to get feedback on something. I think more rules needs to be implemented to be able to get some valueable feedback. Un-structured feedback trough Reddit. Evaluation of github bot as platform for feedback. Evaluation of implemented rules. Evaluate how often the rule gets triggered using a built software system / vs the noise it creates. Evaluate impact on decisions - e.g a commit introduced some inheritance, and later it was found out that inheritance caused trouble. Simulation on other projects to find bugs and false-positives. Used it to further improve the rules and reduce the noise.

Gather a group of Kotlin/Android entusiasts for a workshop where one could order pizza and code using this tool. Focus on feedback. Could for example analyze some code that they know and give some feedback on whether it is useful or not. What would have been useful?

Not that much focus on creating and hitting bullseye with the product, more about the feedback itself and how to respond to it.

Multiple evaluation methods used on different stages of the project: Evaluation running the program on others code, evaluation using a workshop with focus on feedback and usage, evaluation of the initial project using reddit etc.

However, i posted an image of the MVP on Reddit/Kotlin, and it quickly got so many upvotes that it was top this week in under 12 hours. 

\subsection{Communication}


My needs for this solution. Needs a AST to analysis and a framework to help me with this. Need a way of creating a report of the analysis and feed it to something else that can post on github. 


\section{Results}

\subsection{Workshop}
Who is the participants, some background information on them. What did the participants say. What is the takeaway from the workshop? What did i change in the product after having the workshop?

What is the goal, how should i reach the goal of the workshop. Keep the workshop structured.

\subsection{}

\subsection{Short recap of the development process}
Creating a vertical prototype with a simple rule (usecomposition rule) to see if this project it technically possible within the time frame of a masters thesis. 



\subsection{Prototypes}
\subsubsection{Vertical prototype}
Vertical prototype with a few rules implemented.


\subsubsection{Horizontal prototype}
Horizontal prototypes with github PR samples with bot commented. Easily created by pushing code to repo and logging in as the bot. Should create samples that demonstrates possible false positives as well as true positives. Could add much context (helping the developer a lot) or short and concise messages.


\subsection{Detekt-hint specification}
Specification of all the rules.

\subsection{Architecture}
A diagram of how everything interacts with each other.

\subsubsection{Detekt}
Detekt is a static analysis tool for kotlin. Which provides access to a framework for rule creating for analyzing code and reporting warnings.

Detekt gives us access to IntelliJ PSI for analyzing Kotlin code. The PSI is built on top of the \gls{ast} provided by the kotlin compiler, and provides utility methods for modifying and querying of the underlying \gls{ast}.

Type resolution?

\subsubsection{Detekt-hint}

\subsubsection{Danger}

\subsubsection{Danger detekt plugin}

\section{Discussion}
\label{discussion}

\section{Future work}
Implement more rules.
Reducing the amount of false positives.

Providing more context and/or possible solutions in the comments. Images or graphs explaining.
Increasing the performance.
Easing the setup process. Github app? Direct integration with the repository with no setup. A tool like code-climate that does not need any configuration. Plus of tool like this: all setup is in source control. Downside: could be tedious to set up or not all team members want to use it.

\section{Conclusion}
\label{conclusion}

\section{Acknowledgements}
\label{acknowledgements}
Thanks to supervisor, all participants in workshop and prototype sessions.

\printbibliography

\appendix
\label{appendix}
\clearpage
\section{Appendix}

\subsection{Horizontal prototype}
\label{horizontal-prototype}

\begin{figure}[h!]
    \centering
    \includegraphics[width=\textwidth]{../comment_isp.png}
    \caption{Prototype of the \gls{isp} rule }
    \label{fig:isp}
\end{figure}


\begin{figure}[h!]
    \centering
    \includegraphics[width=\textwidth]{../comment_lackOfCohesion.png}
    \caption{Prototype of the \gls{lcom} rule}
    \label{fig:lcom}
\end{figure}


\begin{figure}[h!]
    \centering
    \includegraphics[width=\textwidth]{../comment_ocp.png}
    \caption{Prototype of the \gls{ocp} rule}
    \label{fig:ocp}
\end{figure}


\begin{figure}[h!]
    \centering
    \includegraphics[width=\textwidth]{../demo.png}
    \caption{Prototype of the composition over inheritance / \gls{lsp} rule}
    \label{fig:liskov}
\end{figure}

\clearpage

\subsection{Horizontal prototype schema}
\subsubsection*{Participant number:} What is the number of the participant.
\subsubsection*{Background:} What is the participants background. Experience with software architecture? Knows and uses design principles? Experience with Kotlin?
\subsubsection*{Presentation of rules - For each rule}
\begin{enumerate}
    \item Present the rule. 
    \item Make sure the participant understands the importance of the rule.
    \item When will the rule give a warning?
    \item When will the rule incorrectly give a warning? Will it report false-positives too often? Suggestions on how to reduce the amount?
    \item How much context is needed? Shorter or longer comments? Should include suggestions on possible solutions? Is the comment understandable? Something missing?
\end{enumerate}

\subsubsection*{Other} 
- When reviewing code, what do you think is tedious, and could it be automated?
- Are there any rules/principles missing?

\clearpage
\subsection{Horizontal prototype results}
\textbf{Participant number:} 1 \newline
\textbf{Background:} Studies computer science at \gls{ntnu} with a specialization in computers and systems software. Has experience with developing apps for iOS and web and back-end development. Interested in Software Architecture and writing software of high quality. \\
\textbf{Experience with design principles:} Some \\\\

\noindent UseCompositionInsteadOfInheritance: Could be useful, but potentially have too many false positives. For testing Eirik often creates Mock objects that inherits from the class he wants to Mock, and then overrides methods. Eirik think there is too few cases where this rule will be useful. Suggestions: Reduce the amount of positives by disabling checks for classes with names; Mock. User could specify which class names or a pattern to ignore. Should revisit sentence number two about composition, it could be misleading.\\\\

\noindent LCOM1: Very useful because calculating such a value is'nt something you do while coding. Positive that you can change the threshold of the rule. Suggestion: Which fields and methods could i extract? A comment that suggests a solution. \\\\

\noindent LCOM2: Look more at further analysis to find out what can be extracted. Look into dependencies between function calls as well. Diagram looks cool, but does not give any more value than some plain text explaining what can be extracted. \\\\

\noindent OCP: Somewhat useful. Should have a more specific comment saying if you are doing enum switching or instanceOf checking. \\\\

\noindent ISP: Could be useful. Suggestion to count number of usages of calls in the interface to see which method calls that is not used by any of the classes that implement the interface. \\\\

\noindent Other: Tools for detecting complicated expressions that can be extracted out as a separate method with a descriptive name. Blocks of code should be extracted out as separate methods so that lines of code that belongs together has its own scope. 
\clearpage
\subsection{Workshop schema}
\label{workshop-schema}


\subsection{Pre-study}
\todo{Add pre study as appendix}

\end{document}