\clearpage
\chapter{Appendix}

\section{Horizontal prototype}
\label{horizontal-prototype-images}

\begin{figure}[h!]
    \centering
    \includegraphics[width=\textwidth]{../images/comment_isp.png}
    \caption{Screenshot of the horizontal prototype showing the \gls{isp} rule. It has detected an empty method, which is a sign of violating the \gls{isp}. In this case the \texttt{EconomicPrinter} implements methods from \texttt{AllInOnePrinter} which it does not need. A solution would be to define separate interfaces for each of the responsibilities (e.g \texttt{Printable, Faxable, Scanable}) and let the concrete implementations of printers implement the interfaces they need.}
    \label{fig:isp}
\end{figure}


\begin{figure}[h!]
    \centering
    \includegraphics[width=\textwidth]{../images/comment_lackOfCohesion.png}
    \caption{Screenshot of the horizontal prototype showing the \gls{lcom} rule, with a visual representation of the lack of cohesion. The figure shows which fields that are referenced from each of the methods in the class. In this case all the methods of the class references their own separate field. This indicates that each of the methods and corresponding fields have separate responsibilities within the class. This would therefore be an indication of violating the \gls{srp}.}
    \label{fig:lcom}
\end{figure}


\begin{figure}[h!]
    \centering
    \includegraphics[width=\textwidth]{images/comment_ocp2.png}
    \caption{Screenshot of the horizontal prototype that shows the \gls{ocp} rule, using a simple program for drawing shapes. In this case the rule detected checking of concrete implementations to control flow. The rule suggests creating an abstraction for \texttt{Rectangle}, \texttt{Circle} (e.g an interface \texttt{Shape} with a \texttt{draw} method that all \texttt{Rectangle} and \texttt{Cirle} should implement) such that eventual new shapes added to the program would not need to modify existing code. }
    \label{fig:ocp}
\end{figure}


\begin{figure}[h!]
    \centering
    \includegraphics[width=\textwidth]{images/horizontal-prototype-coh.png}
    \caption{Screenshot of the horizontal prototype that shows the \gls{coi} rule. The rule suggests the use of composition instead of inheritance, and helps testing if the classes adheres to the \gls{lsp}. In this case, the classical Square - Rectangle problem is presented. \texttt{Square} should not be derived from \texttt{Rectangle} as it would violate the \gls{lsp}. \texttt{Square} does not functionally behave like \texttt{Rectangle} as Squares by definition have two equally long sides. \texttt{Rectangle} should have two independent methods for changing its size, but clearly these methods is not appropriate for the \texttt{Square}. }
    \label{fig:liskov}
\end{figure}

\clearpage

\section{Semi-structured interview schema}
\label{semi-structured-interview-schema}
\subsubsection*{Participant number:} What is the number of the participant.
\subsubsection*{Background:} What is the participants background. Experience with software architecture? Knows and uses design principles? Experience with Kotlin?
\subsubsection*{Presentation of rules - For each rule}
\begin{enumerate}
    \item Present the rule. 
    \item Make sure the participant understands the importance of the rule.
    \item When will the rule give a warning?
    \item When will the rule incorrectly give a warning? Will it report false-positives too often? Suggestions on how to reduce the amount?
    \item How much context is needed? Shorter or longer comments? Should include suggestions on possible solutions? Is the comment understandable? Something missing?
\end{enumerate}

\subsubsection*{Other} 
- When reviewing code, what do you think is tedious, and could it be automated?
- Are there any rules/principles missing?

\clearpage
\section{Semi-structured interview results}
\label{horizontal-prototype-interview-results}

\textbf{Participant number:} 1 \newline
\textbf{Background:} Studies computer science at \gls{ntnu} with a specialization in computers and systems software. Has experience with developing apps for iOS and web and back-end development. Interested in Software Architecture and writing software of high quality. \\
\textbf{Experience with design principles:} Some \\\\

\noindent \gls{coi}: Could be useful, but potentially have too many false positives. For testing Eirik often creates Mock objects that inherits from the class he wants to Mock, and then overrides methods. Eirik think there is too few cases where this rule will be useful. Suggestions: Reduce the amount of positives by disabling checks for classes with names; Mock. User could specify which class names or a pattern to ignore. Should revisit sentence number two about composition, it could be misleading.\\\\

\noindent LCOM1: Very useful because calculating such a value is'nt something you do while coding. Positive that you can change the threshold of the rule. Suggestion: Which fields and methods could i extract? A comment that suggests a solution. \\\\

\noindent LCOM2 (With refactoring visualization): Look more at further analysis to find out what can be extracted. Look into dependencies between function calls as well. Diagram looks cool, but does not give any more value than some plain text explaining what can be extracted. \\\\

\noindent \gls{ocp}: Somewhat useful. Should have a more specific comment saying if you are doing enum switching or instanceOf checking. \\\\

\noindent \gls{isp}: Could be useful. Suggestion to count number of usages of calls in the interface to see which method calls that is not used by any of the classes that implement the interface. \\\\

\noindent Other: Tools for detecting complicated expressions that can be extracted out as a separate method with a descriptive name. Blocks of code should be extracted out as separate methods so that lines of code that belongs together has its own scope. 
\clearpage


% Netlight
\noindent\textbf{Participant number:} 2 \newline
\textbf{Background:} Works as a software developer for Netlight, 2 years of professional developer experience. Experienced with Kotlin development and with architecture and design of software systems.\\
\textbf{Experience with design principles:} Yes \\\\

\noindent \gls{coi}: Positive about the rule, but is concerned about not showing warnings when deriving from third party libraries. This is a case where one also should think about using composition instead of inheritance. Suggests to remove this logic, and instead provide configuration options for packages that should not be reported as violations when derived from.  \\\\

\noindent \gls{lcom}: Nothing special. \\\\

\noindent \gls{ocp}: Useful for both instance of checking and enum switching. Enums in Kotlin are powerful, so switching on them is in many cases not needed and polymorphism is used instead.\\\\

\noindent \gls{isp}: Could be useful. May need to handle TODO's specially. \\\\

\noindent Other: In general positive to the tool, and think it has potential. Good that it is easy to ignore warnings, with just a click. It needs more rules before considering using it. For example including detection of Java antipatterns and ensuring that Kotlin code is idiomatic. For example static methods and places where data-classes could be used. The tool could be used as "training wheels" in a team, where one gradually could disable more rules to not create unnecessary noise in the development.

\clearpage
\section{Final prototype}
\label{final-artifact}
\begin{figure}[h!]
    \centering
    \includegraphics[width=\textwidth]{images/final_coh.png}
    \caption{Screenshot of the final prototype that shows the \gls{coi} rule. The rule suggests the use of composition instead of inheritance, and helps testing if the classes adheres to the \gls{lsp}. In this case, the classical Square - Rectangle problem is presented. \texttt{Square} should not be derived from \texttt{Rectangle} as it would violate the \gls{lsp}. \texttt{Square} does not functionally behave like \texttt{Rectangle} as Squares by definition have two equally long sides. \texttt{Rectangle} should have two independent methods for changing its size, but clearly these methods is not appropriate for the \texttt{Square}.}
\end{figure}

\begin{figure}[h!]
    \centering
    \includegraphics[width=\textwidth]{images/final_ocp.png}
    \caption{Screenshot of the final prototype that shows the \gls{ocp} rule, using a simple program for drawing shapes. In this case the rule detected checking of concrete implementations to control flow. The rule suggests creating an abstraction for \texttt{Rectangle}, \texttt{Circle} (e.g an interface \texttt{Shape} with a \texttt{draw} method that all \texttt{Rectangle} and \texttt{Cirle} should implement) such that eventual new shapes added to the program would not need to modify existing code.}
\end{figure}

\begin{figure}[h!]
    \centering
    \includegraphics[width=\textwidth]{images/final_lcom.png}
    \caption{Screenshot of the final prototype showing the \gls{lcom} rule. This is an indication of violating the \gls{srp}, due to low cohesion in the class. In this case, each of the methods reference their own field, telling us that there is no relationship between the different methods, and that they don't need to exist in the same class.}
\end{figure}

\begin{figure}[h!]
    \centering
    \includegraphics[width=\textwidth]{images/final_isp.png}
    \caption{Screenshot of the final prototype showing the \gls{isp} rule. It has detected an empty method, which is a sign of violating the \gls{isp}. In this case the \texttt{EconomicPrinter} implements methods from \texttt{AllInOnePrinter} which it does not need. A solution would be to define separate interfaces for each of the responsibilities (e.g \texttt{Printable, Faxable, Scanable}) and let the concrete implementations of printers implement the interfaces they need.}
\end{figure}
